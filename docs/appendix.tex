\documentclass[letterpaper]{article}
\usepackage{aaai26}
\usepackage{times}
\usepackage{helvet}
\usepackage{courier}
\usepackage[hyphens]{url}
\usepackage{graphicx}
\usepackage{amsmath}
\usepackage{amsfonts}
\usepackage{amssymb}
\usepackage{booktabs}
\usepackage{multirow}
\usepackage{subcaption}
\usepackage{longtable}

\nocopyright

\title{Chain-of-Influence: Tracing Interdependencies Across Time and Features in Clinical Predictive Modeling\\
\large{\textbf{Appendix: Dataset Characteristics and Technical Details}}}

\author{
    Anonymous Submission\\
    \texttt{anonymous@email.com}
}

\begin{document}

\maketitle

\section{Dataset Characteristics}

\subsection{Chronic Kidney Disease Dataset Statistics}

\subsubsection{Cohort Overview and Comparative Analysis}

The Chronic Kidney Disease (CKD) dataset represents a comprehensive longitudinal study of 1,422 patients with progressive kidney disease, followed over 24 months. Table \ref{tab:ckd_comprehensive} presents a comparative analysis between patients who progressed to ESRD versus those who did not, providing critical insights into risk factors and disease progression patterns.

\begin{table}[htbp]
\centering
\caption{CKD Dataset: Comprehensive Patient Characteristics by ESRD Progression Status}
\label{tab:ckd_comprehensive}
\small
\begin{tabular}{@{}lcccc@{}}
\toprule
\textbf{Characteristic} & \textbf{ESRD Progression} & \textbf{No ESRD Progression} & \textbf{P-value} \\
 & \textbf{(n=86, 6.0\%)} & \textbf{(n=1,336, 94.0\%)} & \\
\midrule
\multicolumn{4}{l}{\textbf{Demographics}} \\
Age (years) & $69.13 \pm 12.37$ & $72.04 \pm 11.25$ & $<0.001$ \\
Female & $40$ (46.5\%) & $721$ (54.0\%) & $0.215$ \\
Race & & & $<0.001$ \\
\quad White & $70$ (81.4\%) & $1,242$ (93.0\%) & \\
\quad African American & $12$ (14.0\%) & $60$ (4.5\%) & \\
\quad Other & $4$ (4.6\%) & $34$ (2.5\%) & \\
BMI (kg/m²) & $28.40 \pm 5.32$ & $26.40 \pm 6.20$ & $<0.001$ \\
\midrule
\multicolumn{4}{l}{\textbf{Comorbidities}} \\
Diabetes & $63$ (73.3\%) & $788$ (59.0\%) & $0.009$ \\
Hypertension & $85$ (99.0\%) & $1,323$ (99.0\%) & $0.863$ \\
Cardiovascular Disease & $10$ (17.2\%) & $177$ (18.2\%) & $0.990$ \\
Anemia & $55$ (64.0\%) & $828$ (62.0\%) & $0.714$ \\
Metabolic Acidosis & $22$ (25.6\%) & $240$ (18.0\%) & $0.077$ \\
Proteinuria & $11$ (12.8\%) & $227$ (17.0\%) & $0.312$ \\
Secondary Hyperparathyroidism & $28$ (32.6\%) & $240$ (18.0\%) & $<0.001$ \\
Phosphatemia & $4$ (4.7\%) & $40$ (3.0\%) & $0.390$ \\
Atherosclerosis & $6$ (9.4\%) & $149$ (14.6\%) & $0.320$ \\
Heart Failure & $6$ (7.0\%) & $120$ (9.0\%) & $0.526$ \\
Stroke & $1$ (1.2\%) & $40$ (3.0\%) & $0.506$ \\
Conduction \& Dysrhythmias & $4$ (4.7\%) & $214$ (16.0\%) & $0.005$ \\
Myocardial Infarction & $31$ (51.7\%) & $316$ (32.4\%) & $0.003$ \\
Fluid \& Electrolyte Disorders & $9$ (17.6\%) & $122$ (13.3\%) & $0.503$ \\
Metabolic Disorders & $5$ (9.4\%) & $60$ (6.5\%) & $0.581$ \\
Nutritional Disorders & $6$ (10.2\%) & $106$ (11.6\%) & $0.900$ \\
CKD Stage 4 & $47$ (54.7\%) & $298$ (22.3\%) & $<0.001$ \\
CKD Stage 5 & $42$ (48.8\%) & $277$ (20.7\%) & $<0.001$ \\
\bottomrule
\end{tabular}
\end{table}

\subsubsection{Feature Categories and Descriptions}

The 38 clinical features in the CKD dataset are systematically organized across four major categories, each capturing distinct aspects of patient health and disease progression.

\begin{longtable}{@{}llp{8cm}@{}}
\caption{CKD Dataset: Complete Feature Specifications} \label{tab:ckd_features} \\
\toprule
\textbf{Category} & \textbf{Feature} & \textbf{Description} \\
\midrule
\endfirsthead
\multicolumn{3}{c}{\tablename\ \thetable\ -- continued from previous page} \\
\toprule
\textbf{Category} & \textbf{Feature} & \textbf{Description} \\
\midrule
\endhead
\midrule
\multicolumn{3}{r}{Continued on next page} \\
\endfoot
\bottomrule
\endlastfoot

\multirow{4}{*}{\textbf{Demographics}} 
& Age & Patient age at baseline (years) \\
& Gender & Binary indicator (Male=0, Female=1) \\
& Race & Categorical encoding (White, Black, Hispanic, Other) \\
& BMI & Body Mass Index (kg/m²) \\
\midrule

\multirow{18}{*}{\textbf{Comorbidities}} 
& Diabetes & Type 1 or Type 2 diabetes mellitus diagnosis \\
& Htn & Hypertension (systolic ≥140 or diastolic ≥90 mmHg) \\
& Cvd & Cardiovascular disease (coronary, cerebrovascular, peripheral) \\
& Anemia & Hemoglobin <13 g/dL (men) or <12 g/dL (women) \\
& MA & Mineral and bone disorders \\
& Prot & Proteinuria (>300 mg/day or >300 mg/g creatinine) \\
& SH & Secondary hyperparathyroidism \\
& Phos & Phosphorus disorders (hyperphosphatemia) \\
& Athsc & Atherosclerosis \\
& CHF & Congestive heart failure \\
& Stroke & Cerebrovascular accident history \\
& CD & Coronary disease \\
& MI & Myocardial infarction history \\
& FE & Iron deficiency \\
& MD & Metabolic disorders \\
& ND & Nutritional deficiency \\
& S4 & CKD Stage 4 (eGFR 15-29 mL/min/1.73m²) \\
& S5 & CKD Stage 5 (eGFR <15 mL/min/1.73m²) \\
\midrule

\multirow{8}{*}{\textbf{Lab Biomarkers}} 
& Serum\_Calcium & Serum calcium levels (mg/dL) \\
& eGFR & Estimated glomerular filtration rate (mL/min/1.73m²) \\
& Phosphorus & Serum phosphorus levels (mg/dL) \\
& Intact\_PTH & Intact parathyroid hormone (pg/mL) \\
& Hemoglobin & Blood hemoglobin concentration (g/dL) \\
& UACR & Urine albumin-to-creatinine ratio (mg/g) \\
& Bicarbonate & Serum bicarbonate levels (mEq/L) \\
\midrule

\multirow{8}{*}{\textbf{Healthcare Utilization}} 
& n\_claims\_DR & Number of durable medical equipment/drug claims \\
& n\_claims\_I & Number of inpatient claims \\
& n\_claims\_O & Number of outpatient claims \\
& n\_claims\_P & Number of physician service claims \\
& net\_exp\_DR & Net expenditure for DME/drugs (\$) \\
& net\_exp\_I & Net expenditure for inpatient services (\$) \\
& net\_exp\_O & Net expenditure for outpatient services (\$) \\
& net\_exp\_P & Net expenditure for physician services (\$) \\

\end{longtable}

\begin{table}[htbp]
\centering
\caption{CKD Dataset: Laboratory Values and Healthcare Utilization by ESRD Progression Status}
\label{tab:ckd_clinical_claims}
\small
\begin{tabular}{@{}lcccc@{}}
\toprule
\textbf{Feature} & \textbf{ESRD Progression} & \textbf{No ESRD Progression} & \textbf{P-value} \\
 & \textbf{(n=86)} & \textbf{(n=1,336)} & \\
\midrule
\multicolumn{4}{l}{\textbf{Laboratory Biomarkers}} \\
eGFR (mL/min/1.73m²) & $17.21 \pm 5.46$ & $22.78 \pm 5.66$ & $<0.001$ \\
Hemoglobin (g/dL) & $12.15 \pm 2.19$ & $14.25 \pm 1.80$ & $<0.001$ \\
Bicarbonate (mEq/L) & $22.90 \pm 6.36$ & $25.30 \pm 4.22$ & $0.001$ \\
Serum Calcium (mg/dL) & $9.39 \pm 3.62$ & $10.21 \pm 2.86$ & $0.042$ \\
Phosphorus (mg/dL) & $3.61 \pm 0.87$ & $3.52 \pm 0.72$ & $0.350$ \\
Intact PTH (pg/mL) & $78.66 \pm 40.23$ & $62.72 \pm 37.32$ & $0.001$ \\
\midrule
\multicolumn{4}{l}{\textbf{Healthcare Utilization Claims}} \\
Pharmacy Claims (count) & $120 \pm 94$ & $109 \pm 86$ & $0.293$ \\
Inpatient Claims (count) & $3.85 \pm 3.41$ & $3.74 \pm 3.62$ & $0.773$ \\
Outpatient Claims (count) & $27.78 \pm 24.75$ & $22.07 \pm 19.13$ & $0.039$ \\
Professional Claims (count) & $105.37 \pm 77.56$ & $87.43 \pm 68.02$ & $0.039$ \\
\midrule
\multicolumn{4}{l}{\textbf{Healthcare Expenditures (\$)}} \\
Pharmacy Expenditure & $12,053 \pm 11,596$ & $10,440 \pm 20,662$ & $0.242$ \\
Inpatient Expenditure & $33,909 \pm 53,540$ & $29,440 \pm 32,541$ & $0.446$ \\
Outpatient Expenditure & $9,354 \pm 17,522$ & $8,554 \pm 17,492$ & $0.682$ \\
Professional Expenditure & $15,512 \pm 18,657$ & $11,640 \pm 12,748$ & $0.061$ \\
\bottomrule
\end{tabular}
\end{table}

\subsubsection{Statistical Analysis and Clinical Significance}

\textbf{Cohort Characteristics:} The complete cohort comprises 1,422 CKD patients with 24-month follow-up, yielding 8 temporal observations per patient (3-month intervals). ESRD progression occurred in 86 patients (6.0\%), representing a clinically realistic progression rate for this advanced CKD population.

\textbf{Significant Risk Factors (p < 0.05):} Statistical analysis revealed several key differentiators between ESRD progressors and non-progressors:

\textit{Demographic Factors:}
\begin{itemize}
    \item Younger age (69.1 vs 72.0 years) - counterintuitive finding suggesting more aggressive disease in younger patients
    \item Higher BMI (28.4 vs 26.4 kg/m²) - indicating metabolic burden
    \item Higher proportion of African Americans (14.0\% vs 4.5\%) - known ethnic disparity in CKD progression
\end{itemize}

\textit{Comorbidity Profile:}
\begin{itemize}
    \item Higher diabetes prevalence (73.3\% vs 59.0\%) - primary driver of CKD progression
    \item Increased secondary hyperparathyroidism (32.6\% vs 18.0\%) - marker of mineral bone disorder
    \item More myocardial infarction (51.7\% vs 32.4\%) - cardiovascular complications
    \item Paradoxically lower conduction disorders (4.7\% vs 16.0\%) - may reflect survival bias
    \item Higher CKD stage 4 (54.7\% vs 22.3\%) and stage 5 (48.8\% vs 20.7\%) prevalence
\end{itemize}

\textit{Laboratory Markers:}
\begin{itemize}
    \item Significantly lower eGFR (17.2 vs 22.8 mL/min/1.73m²) - expected primary marker
    \item Lower hemoglobin (12.2 vs 14.3 g/dL) - indicating CKD-related anemia
    \item Lower bicarbonate (22.9 vs 25.3 mEq/L) - metabolic acidosis marker
    \item Higher intact PTH (78.7 vs 62.7 pg/mL) - secondary hyperparathyroidism
    \item Lower serum calcium (9.4 vs 10.2 mg/dL) - mineral bone disorder
\end{itemize}

\textit{Healthcare Utilization:}
\begin{itemize}
    \item Higher outpatient claims (27.8 vs 22.1) - increased monitoring needs
    \item Higher professional claims (105.4 vs 87.4) - specialist consultations
\end{itemize}

\subsubsection{Temporal Patterns and Data Quality}

\textbf{Longitudinal Structure:} The dataset provides 24 months of follow-up with standardized 3-month assessment intervals, resulting in 8 temporal observation points per patient. This temporal resolution is optimal for capturing the gradual progression patterns characteristic of chronic kidney disease.

\textbf{Missing Data Handling:} Laboratory values exhibit expected patterns of clinical missingness, with more frequent monitoring in patients with advanced CKD stages. Our preprocessing pipeline handles missing data through:
\begin{enumerate}
    \item Forward-fill within patient timelines
    \item Linear interpolation for short gaps
    \item Median imputation stratified by CKD stage for remaining missing values
\end{enumerate}

\textbf{Clinical Validation:} The observed progression rate (6.0\%) and risk factor profile align with established nephrology literature, validating the dataset's clinical authenticity and generalizability to real-world CKD populations.

\subsection{MIMIC-IV Dataset Characteristics}

\subsubsection{Cohort Construction and Selection Criteria}

The MIMIC-IV cohort was constructed through a systematic pipeline designed to ensure data quality and clinical relevance for mortality prediction tasks.

\textbf{Initial Database Statistics:}
\begin{itemize}
    \item Total ICU stays in MIMIC-IV: 94,458
    \item Unique patients: 65,366
    \item Multiple ICU stays per patient: 29,092 additional stays
\end{itemize}

\textbf{Cohort Construction:}
Our preprocessing pipeline selects the first ICU stay per patient to ensure independence of observations, resulting in exactly 65,366 patients. This approach eliminates potential confounding from repeated admissions while maintaining a substantial sample size for robust model training and evaluation.

\textbf{Final Cohort:} 65,366 patients meeting all inclusion criteria.

\subsubsection{Feature Extraction and Processing}

Table \ref{tab:mimic_features} provides detailed specifications for all 15 features extracted from MIMIC-IV.

\begin{longtable}{@{}llp{6cm}l@{}}
\caption{MIMIC-IV Dataset: Complete Feature Specifications} \label{tab:mimic_features} \\
\toprule
\textbf{Category} & \textbf{Feature} & \textbf{Description} & \textbf{MIMIC Item IDs} \\
\midrule
\endfirsthead
\multicolumn{4}{c}{\tablename\ \thetable\ -- continued from previous page} \\
\toprule
\textbf{Category} & \textbf{Feature} & \textbf{Description} & \textbf{MIMIC Item IDs} \\
\midrule
\endhead
\midrule
\multicolumn{4}{r}{Continued on next page} \\
\endfoot
\bottomrule
\endlastfoot

\multirow{7}{*}{\textbf{Vital Signs}} 
& Heart Rate & Beats per minute & 220045 \\
& Systolic BP & Systolic blood pressure (mmHg) & 220179 \\
& Diastolic BP & Diastolic blood pressure (mmHg) & 220180 \\
& Mean BP & Mean arterial pressure (mmHg) & 220181 \\
& Respiratory Rate & Breaths per minute & 220210 \\
& Temperature & Body temperature (°C) & 223761, 678 \\
& SpO2 & Oxygen saturation (\%) & 220277 \\
\midrule

\multirow{6}{*}{\textbf{Laboratory Tests}} 
& Creatinine & Serum creatinine (mg/dL) & 50912 \\
& Glucose & Blood glucose (mg/dL) & 50931 \\
& Sodium & Serum sodium (mEq/L) & 50983 \\
& Potassium & Serum potassium (mEq/L) & 50971 \\
& Hematocrit & Hematocrit (\%) & 51221 \\
& WBC & White blood cell count (K/µL) & 51301 \\
\midrule

\multirow{2}{*}{\textbf{Demographics}} 
& Gender & Binary (Male=0, Female=1) & - \\
& Age & Age at admission (years) & - \\

\end{longtable}

\subsubsection{Comprehensive Feature Statistics}

Table \ref{tab:mimic_detailed_stats} presents detailed statistics for all MIMIC-IV features, including percentile distributions to capture the full range of physiological values encountered in critical care.

\begin{longtable}{@{}lcccc@{}}
\caption{MIMIC-IV Dataset: Comprehensive Feature Statistics (n=65,366 patients)} \label{tab:mimic_detailed_stats} \\
\toprule
\textbf{Feature} & \textbf{Mean ± SD} & \textbf{Median [IQR]} & \textbf{Missing (\%)} \\
\midrule
\endfirsthead
\multicolumn{4}{c}{\tablename\ \thetable\ -- continued from previous page} \\
\toprule
\textbf{Feature} & \textbf{Mean ± SD} & \textbf{Median [IQR]} & \textbf{Missing (\%)} \\
\midrule
\endhead
\midrule
\multicolumn{4}{r}{Continued on next page} \\
\endfoot
\bottomrule
\endlastfoot

\multicolumn{4}{c}{\textbf{Demographics}} \\
\midrule
Age (years) & $64.5 \pm 17.1$ & $66.0 [54.0, 78.0]$ & $0.0$ \\
Gender (Female \%) & $43.8\%$ & - & $0.0$ \\
\midrule
\multicolumn{4}{c}{\textbf{Vital Signs}} \\
\midrule
Heart Rate (bpm) & $84.1 \pm 18.9^*$ & $82.0 [71.0, 95.0]$ & $27.5$ \\
Systolic BP (mmHg) & $119.9 \pm 25.8^*$ & $117.0 [104.0, 133.0]$ & $51.6$ \\
Diastolic BP (mmHg) & $67.4 \pm 15.8^*$ & $64.0 [55.0, 75.0]$ & $51.6$ \\
Mean BP (mmHg) & $87.6 \pm 16.9^*$ & $78.0 [69.0, 89.0]$ & $51.6$ \\
Respiratory Rate (rpm) & $22.7 \pm 6.8^*$ & $19.0 [15.0, 22.0]$ & $28.4$ \\
Temperature (°C) & $37.0 \pm 0.8$ & $36.8 [36.6, 37.2]$ & $81.3$ \\
SpO2 (\%) & $96.8 \pm 3.2^*$ & $97.0 [95.0, 99.0]$ & $28.9$ \\
\midrule
\multicolumn{4}{c}{\textbf{Laboratory Tests}} \\
\midrule
Creatinine (mg/dL) & $1.4 \pm 1.5$ & $1.0 [0.7, 1.5]$ & $93.1$ \\
Glucose (mg/dL) & $141.1 \pm 69.7$ & $125.0 [103.0, 157.0]$ & $93.5$ \\
Sodium (mEq/L) & $138.2 \pm 5.6$ & $138.0 [135.0, 141.0]$ & $92.9$ \\
Potassium (mEq/L) & $4.2 \pm 0.7$ & $4.1 [3.8, 4.5]$ & $92.8$ \\
Hematocrit (\%) & $31.1 \pm 6.0$ & $30.5 [26.7, 35.1]$ & $92.3$ \\
WBC (K/µL) & $12.3 \pm 10.7$ & $10.7 [7.8, 14.7]$ & $93.3$ \\

\end{longtable}

\textit{Note: $^*$ indicates features with outliers present in raw data that require preprocessing. Standard deviations shown are after outlier detection but before winsorization.}

\subsubsection{Data Quality and Preprocessing Pipeline}

\textbf{Missing Data Handling:}
\begin{enumerate}
    \item \textbf{Forward Fill:} Within each patient's 48-hour window, missing values are forward-filled from the last available measurement
    \item \textbf{Backward Fill:} If no prior measurement exists, backward-fill from the next available measurement  
    \item \textbf{Linear Interpolation:} For gaps between measurements, linear interpolation is applied
    \item \textbf{Median Imputation:} Remaining missing values are imputed using feature-specific medians computed from the training set
\end{enumerate}

\textbf{Outlier Detection and Treatment:}
Physiologically implausible values are identified using clinical knowledge-based thresholds and statistical methods:
\begin{itemize}
    \item Heart Rate: <20 or >220 bpm
    \item Blood Pressure: Systolic <50 or >250 mmHg, Diastolic <20 or >150 mmHg
    \item Temperature: <32°C or >42°C  
    \item SpO2: <70\% or >100\%
    \item Laboratory values: Beyond 99.5th percentile of physiologically reasonable ranges
\end{itemize}

Outliers are winsorized to the 1st and 99th percentiles to preserve distributional properties while reducing the impact of measurement errors.

\textbf{Normalization:}
All continuous features are standardized using z-score normalization: $z = \frac{x - \mu}{\sigma}$, where $\mu$ and $\sigma$ are computed on the training set only to prevent data leakage.

\subsubsection{Cohort Characteristics and Outcomes}

\begin{table}[htbp]
\centering
\caption{MIMIC-IV Cohort: Clinical Characteristics and Outcomes}
\label{tab:mimic_cohort}
\begin{tabular}{@{}lcc@{}}
\toprule
\textbf{Characteristic} & \textbf{Value} & \textbf{Percentage} \\
\midrule
Total Patients & 65,366 & - \\
Age Distribution & & \\
\quad Mean Age & $64.5 \pm 17.1$ years & - \\
\quad Age Range & 18-103 years & - \\
Gender & & \\
\quad Male & 36,720 & 56.2\% \\
\quad Female & 28,646 & 43.8\% \\
Temporal Structure & & \\
\quad Observation Window & 48 hours & Hourly resolution \\
\quad Total Observations & 3,137,568 & (65,366 × 48) \\
\quad Features per Patient & 15 & 7 vitals + 6 labs + 2 demographics \\
\midrule
\textbf{Clinical Outcomes} & & \\
Hospital Mortality & 7,086 & 10.8\% \\
Survivors & 58,280 & 89.2\% \\
\midrule
\textbf{Data Quality Characteristics} & & \\
Vital Signs Missing Data & 27.5-81.3\% & Variable by feature type \\
Laboratory Tests Missing Data & 92.3-93.5\% & Expected for ICU setting \\
Demographics Missing Data & 0.0\% & Complete for all patients \\
\bottomrule
\end{tabular}
\end{table}

\section{Data Processing Implementation Details}

\subsection{MIMIC-IV Preprocessing Pipeline}

The preprocessing pipeline for MIMIC-IV data involves several computational steps designed to handle the scale and complexity of the raw database:

\textbf{Computational Requirements:}
\begin{itemize}
    \item Memory usage: ~32GB RAM for full dataset processing
    \item Processing time: ~4-6 hours on 16-core CPU
    \item Storage: Raw data (120GB), processed data (8.2GB)
\end{itemize}

\textbf{Batch Processing Strategy:}
To manage memory constraints, patient processing is batched in groups of 1,000 patients, with relevant chartevents and labevents filtered for each batch to minimize memory footprint.

\textbf{Temporal Alignment:}
All timestamps are aligned to ICU admission time (intime), with measurements indexed by hour offset (0-47) from admission. This standardization enables consistent temporal modeling across patients.

\subsection{Feature Engineering Considerations}

\textbf{Clinical Rationale for Feature Selection:}
\begin{itemize}
    \item \textbf{Vital Signs:} Selected based on immediate availability and prognostic value in ICU mortality prediction
    \item \textbf{Laboratory Tests:} Chosen to represent major organ systems (kidney, metabolism, hematology, infection)
    \item \textbf{Exclusions:} Medication and intervention data excluded to focus on physiological state rather than treatment effects
\end{itemize}

\textbf{Temporal Binning Strategy:}
Measurements within each hour are aggregated using median values to reduce noise while preserving clinical trends. This approach balances temporal resolution with data stability.

\section{Experimental Configuration Details}

\subsection{Hardware and Software Environment}

\textbf{Computational Infrastructure:}
\begin{itemize}
    \item GPUs: 4x NVIDIA A100 40GB
    \item CPU: 64-core AMD EPYC 7763
    \item RAM: 256GB DDR4
    \item Storage: 2TB NVMe SSD
\end{itemize}

\textbf{Software Stack:}
\begin{itemize}
    \item Python 3.9.12
    \item PyTorch 1.12.1
    \item NumPy 1.21.6
    \item Pandas 1.4.3
    \item Scikit-learn 1.1.1
    \item CUDA 11.6
\end{itemize}

\subsection{Model Training Configuration}

\textbf{Optimization Settings:}
\begin{itemize}
    \item Optimizer: AdamW with weight decay 1e-4
    \item Learning rate schedule: Cosine annealing with warm restart
    \item Gradient clipping: Maximum norm 1.0
    \item Early stopping: Patience 15 epochs on validation AUROC
\end{itemize}

\textbf{Data Splitting Strategy:}
\begin{itemize}
    \item Training: 60\% (stratified by outcome)
    \item Validation: 20\% (for hyperparameter tuning)
    \item Testing: 20\% (held out for final evaluation)
    \item Cross-validation: 5-fold for hyperparameter search
\end{itemize}

\textbf{Class Imbalance Handling:} 
TSMOTE (Temporal SMOTE) is applied exclusively to training data after the initial split to prevent data leakage. The algorithm:
\begin{enumerate}
    \item Identifies minority class samples in the training set
    \item Computes $k$-nearest neighbors in flattened time series space
    \item Generates synthetic samples via temporal interpolation between neighbors
    \item Adds small Gaussian noise ($\sigma = 0.005$) to maintain temporal realism
    \item Balances the training set while preserving original validation/test distributions
\end{enumerate}

For the CKD dataset, TSMOTE typically generates ~800 additional ESRD progression samples, bringing the training class ratio from 6:94 to approximately 45:55. For MIMIC-IV, the process generates ~5,500 additional mortality cases, achieving similar balance.

\section{Model Interpretability Analysis}

\subsection{Feature Importance Comparison}

Figure \ref{fig:feature_comparison} presents a comprehensive comparison of feature importance rankings between RETAIN and Chain-of-Influence (CoI) models, displayed in descending order of importance scores for both approaches.

\begin{figure}[htbp]
\centering
\includegraphics[width=0.9\textwidth]{feature_importance_comparison.png}
\caption{Feature importance comparison between RETAIN (left) and CoI (right) models. Both models consistently rank diabetes, laboratory values (S4, S5), net expenditure patterns, and hemoglobin among top features. Notable differences include CoI's higher ranking of eGFR (6th vs 14th position) and more balanced importance distribution across features, suggesting broader integration of clinical information.}
\label{fig:feature_comparison}
\end{figure}

The comparative analysis reveals both convergent and divergent patterns in feature importance attribution. Convergent patterns include the consistent high ranking of diabetes mellitus, specific laboratory biomarkers (S4, S5), healthcare utilization metrics (net\_exp\_P/O), and hematological parameters across both models. These commonalities validate the clinical relevance of both approaches in identifying established risk factors for CKD progression.

Divergent patterns provide insights into the architectural differences between the models. CoI's higher ranking of eGFR reflects its cross-feature attention mechanism's ability to capture the central role of kidney function measurements in disease progression cascades. The more balanced importance distribution observed in CoI suggests that its transformer-based architecture integrates information from a broader spectrum of clinical variables, potentially leading to more robust and generalizable predictions.

\end{document} 